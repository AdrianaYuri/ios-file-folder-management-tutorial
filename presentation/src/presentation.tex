\documentclass[serif,mathserif]{beamer}


\usepackage{fleqn}
%%\usepackage{modefs}
%% \usepackage{math}
\usepackage[utf8]{inputenc}
\usepackage{amsmath, amsfonts, epsfig, xspace}
\usepackage{pstricks,pst-node}
\usepackage{multimedia}
\usepackage[normal,tight,center]{subfigure}
\usepackage{listings}
\setlength{\subfigcapskip}{-.5em}
\usepackage{beamerthemesplit}
\renewcommand\sfdefault{phv}
\renewcommand\familydefault{\sfdefault}
\usetheme{default}
\usepackage{color}
\useoutertheme{default}
\usepackage{texnansi}
\usepackage{marvosym}
\definecolor{bottomcolour}{rgb}{0.32,0.3,0.38}
\definecolor{middlecolour}{rgb}{0.08,0.08,0.16}
\setbeamerfont{title}{size=\Huge}
\setbeamercolor{structure}{fg=gray}
\setbeamertemplate{frametitle}[default]%[center]
\setbeamercolor{normal text}{bg=black, fg=white}
\setbeamertemplate{background canvas}[vertical shading]
[bottom=bottomcolour, middle=middlecolour, top=black]
\setbeamertemplate{items}[circle]
\setbeamerfont{frametitle}{size=\huge}
\setbeamertemplate{navigation symbols}{} %no nav symbols

\author[Vinicius Miana]{Vinicius Miana}

\title[Short Title\hspace{2em}\insertframenumber/\inserttotalframenumber]{Tutorial Gerenciamento de Arquivos}

\date{18 de março de 2014} %leave out for today's date to be insterted

\institute{Universidade Presbiteriana Mackenzie}

\begin{document}


\lstset{language=[Objective]C,
  keywordstyle=\color{blue},
  keywords=[2]{NSFileManager,URLsForDirectory, NSDocumentDirectory, NSUserDomainMask, NSTemporaryDirectory, writeToFile},
  keywordstyle=[2]\color{orange},
  stringstyle=\color{red},
  commentstyle=\color{green},
  breaklines=true,
  basicstyle=\footnotesize
}

\maketitle

\section{Gerenciamento de Arquivos}  


\begin{frame}
  \frametitle{Questão Essencial: Como armazenar dados do seu aplicativo? }
   Desafio: Criar um app que armazene receitas de cozinha
   Guiding Questions:
  \begin{itemize}
  \item Qual a estrutura de diretórios do iOS? 
  \item Como acessar os diretórios importantes do seu aplicativo?
  \item Como fazer a manipulação de arquivos?
  \item Como salvar objetos?
  \end{itemize}
\end{frame}


\begin{frame}
  \frametitle{Estrutura de Diretórios}
  \begin{itemize}
  \item Photos
  \item Apps
    \begin{itemize}
      \item Seu App - Pasta raiz do seu aplicativo
        \begin{itemize}
          \item Documents - Conteúdo do usuário - sincronizado via iTunes ou iCloud
          \item Seu.app - O binário do seu aplicativo
          \item tmp - Dados temporários - Pasta acessível através de NSTemporaryDirectory  
          \item Library/Caches - Dados do aplicativo que podem ser recriados se apagado - não sincronizado
          \item Library/Preferences - preferências do aplicativo - configuráveis em Settings e manipuladas pelo NSUserDefaults
          \item Library/Application Support - Dados criadoos pelo aplicativo - tem que ser criado manualmente - sincronizado
        \end{itemize}  
      \item .... - Outros aplicativos
    \end{itemize}    
  \item .... - Outras pastas de sistema
  \end{itemize}
\end{frame}



\begin{frame}[fragile]
  \frametitle{Acessando diretórios}
  Atividades
  \begin{itemize}
  \item Leia a documentação das classes: NSFileManager, URLSForDirectory, NSSearchPathDomainMask, NSSearchPathDirectory 
  \item Corrija, execute e identifique o que faz o código abaixo.
  \end{itemize}
  \begin{lstlisting}    
    NSFileManager *fileManager = [[NSFileManager alloc] init];
    NSArray *urls = [fileManager URLsForDirectory:NSDocumentDirectory inDomains:NSUserDomainMask];
    for(int i = 0; i < [urls count]; i++  ) {
      NSLog(@"%@", urls[0]);
    }
  \end{lstlisting}  
\end{frame}

\begin{frame}[fragile]
  \frametitle{Manipulação de arquivos}
  Para gravar NSString, UIImage, NSDictionary, NSArray, NSNumber, NSDate e NSData temos métodos para gravá-lo e recuperá-lo
  \begin{itemize}
    \item Pesquise na documentação - Quais métodos são estes?
    \item Corrija, execute e identifique o que faz o código abaixo.   
    \item Para pensar, o que fazer com outros tipos de dados, como char, byte, int? (Dica: NSData) 
  \end{itemize}
  \begin{lstlisting}
    NSString *caminho = [NSTemporaryDirectory() stringByAppendingPathComponent:@"MeuArquivo.txt"];
    NSArray *nomes = @[@"Paz", @"Amor"];
     OO sultado = [nomes writeToFile:caminho atomically:YES];
    NSArray *leitura = [[NSArray alloc] initWithContentsOfFile:caminho];
   if ([leitura count] != [nomes count]) NSLog(@"Falha de leitura");
    if(!resultado) NSLog(@"Falha de escrita");
  \end{lstlisting}
\end{frame}

\begin{frame}
  \frametitle{Outras operações}
    \begin{itemize}
     \item Concatenando nomes ao caminho: [NSString stringByAppendingPathComponent:...]
     \item Criando diretórios [NSFileManager createDirectoryAtPath:...]
     \item Listando o conteúdo de um diretório [NSFileManager contentsOfDirectoryAtPath:...]
     \item Removendo [NSFileManager removeItemAtPath:....]
    \end{itemize}
    Mais uma classe para pesquisar: NSURL -> Guiding Question: Como obter a data de criação de um arquivo?
\end{frame}  

\begin{frame}
  \frametitle{Armazenando Objetos}
  \begin{itemize}
  \item Como formatar objetos para gravação em arquivo? (Árvore ou Grafo) -> (Array) 
  \item Marshalling e Unmarshalling - o que é isso?
  \item NSKeyedArchiver e NSKeyedUnarchiver
  \item Como tornar possível o Marshalling e Unmarshalling de objetos? 
  \end{itemize}
  Verificar o protocolo NSCoding
\end{frame}  


\begin{frame}[fragile]
  \frametitle{Usando o protocolo NSCoding}
  Marshalling
  \begin{lstlisting}
    -(void)encodeWithCode:(NSCoder*)aCoder {
      [aCoder encodeObject:atributo forKey:@"atributo"];
      // fazer isso para todos os atributos
    }
  \end{lstlisting}
  Unmarshalling
  \begin{lstlisting}    
    -(id)initWithCoder:(NSCoder*)aDecoder {
      self = [super init];
      if(self) {
         [self setAtributo:[aCoder decodeObjectForKey:@"atributo"]];
         // fazer isso para todos os atributos
      }
      return self;
    }
  \end{lstlisting}  
\end{frame}



\begin{frame}
    \frametitle{Juntando tudo}
    Os objetos podems ser armazenados através e recuperados através de:
    \begin{itemize}
    \item [NSKeyedArchiver archiveRootObject:objeto toFile:pathParaOArquivo];
    \item objeto = [NSKeyedUnarchiver unarchiveObjectWithFile:pathParaOArquivo];
    \end{itemize}
\end{frame}


\begin{frame}
  \frametitle{O desafio - Criar um app que armazena receitas}
  \begin{itemize}
  \item Fork e Clone o projeto do github
  \item BRONZE - Corrija o programa para poder guardar corretamente os ingredientes.
  \item PRATA - Altere o programa para poder guardar várias receitas.
  \item OURO - 
  \end{itemize}
\end{frame}



\begin{frame}
  \frametitle{Questions}
\end{frame}
\end{document}
